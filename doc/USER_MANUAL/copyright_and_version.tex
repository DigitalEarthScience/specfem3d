%%%%%%%%%%%%%%%%%%%%%%%%%%%%%%%%%%%%%%%%%%%%%%%%%

\chapter*{Copyright}
\addcontentsline{toc}{chapter}{Copyright}

Main historical authors: Dimitri Komatitsch and Jeroen Tromp

CNRS, France and Princeton University, USA\newline
$\copyright$ October 2017\newline

\noindent
This program is free software; you can redistribute it and/or modify
it under the terms of the GNU General Public License as published
by the Free Software Foundation (see Appendix \ref{cha:License}).\newline

\noindent
Please note that by contributing to this code, the developer understands and agrees that this project and contribution
are public and fall under the open source license mentioned above.\newline

\noindent
\textbf{\underline{Evolution of the code:}}\newline

version 4.0, March 2023:
Rafael Almada, Jean-Paul Ampuero, Etienne Bachmann, Kangchen Bai, Stephen Beller, Jordan Bishop, Alexis Bottero,
Emanuele Casarotti, Clement Durochat, Rene Gassmoeller, Hom Nath Gharti, Leopold Grinberg, Aakash Gupta,
Foivos Karakostas, Dimitri Komatitsch, Qinya Liu, Geordie McBain, Ryan Modrak, Vadim Monteiller, Masaru Nagaso, Elif Oral,
Daniel Peter, Elliott Sales de Andrade,  James Smith, Carl Tape, Eduardo Valero Cano, Huihui Weng, Zhinan Xie:
various code improvements (fault solver, FK/DSM/AxiSEM coupling, gravity perturbations, energy integrals);
inverse problem module; parallel heuristic mesh decomposer; ADIOS2 file I/O support;
ASDF seismogram output; HIP GPU support.\newline


version 3.0, December 2014: many developers.
Convolutional PML, LDDRK time scheme, bulk attenuation support, simultaneous MPI runs,
ADIOS file I/O support, new seismogram names,
Deville routines for additional GLL degrees, tomography tools, unit/regression test framework,
improved CUDA GPUs performance, additonal GEOCUBIT support, better make compilation,
git versioning system. \newline


version 2.1, July 2012: Max Rietmann, Peter Messmer, Daniel Peter, Dimitri
Komatitsch, Joseph Charles, Zhinan Xie: support for CUDA GPUs, better
CFL stability for the Stacey absorbing conditions. \newline


version 2.0, November 2010: Dimitri Komatitsch, Nicolas Le Goff, Roland
Martin and Pieyre Le Loher, University of Pau, France, Daniel Peter,
Jeroen Tromp and the Princeton group of developers, Princeton University,
USA, and Emanuele Casarotti, INGV Roma, Italy: support for CUBIT meshes
decomposed by SCOTCH; much faster solver using Michel Deville's inlined
matrix products.\newline


version 1.4 Dimitri Komatitsch, University of Pau, Qinya Liu and others,
Caltech, September 2006: better adjoint and kernel calculations, faster
and better I/Os on very large systems, many small improvements and
bug fixes.\newline


version 1.3 Dimitri Komatitsch, University of Pau, and Qinya Liu, Caltech,
July 2005: serial version, regular mesh, adjoint and kernel calculations,
ParaView support.\newline


version 1.2 Min Chen and Dimitri Komatitsch, Caltech, July 2004: full
anisotropy, volume movie.\newline


version 1.1 Dimitri Komatitsch, Caltech, October 2002: Zhu's Moho map,
scaling of $V_{s}$ with depth, Hauksson's regional model, attenuation,
oceans, movies.\newline


version 1.0 (MPI) Dimitri Komatitsch, Caltech, USA, May 2002: first MPI version
based on global code.\newline


Dimitri Komatitsch, IPG Paris, France, December 1996: first 3-D solver
for the CM-5 Connection Machine, parallelized on 128 processors using
Connection Machine Fortran.\newline

